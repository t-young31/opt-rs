\documentclass[10pt]{article}
\usepackage{bm, url, graphicx, amsmath, color}
\usepackage{amsmath}
\usepackage{hyperref}
\nonstopmode

\title{Forcefields}
\author{...}

\begin{document}

\maketitle

\section{UFF}
The theory presented here is adapted from Rappe et. al, JACS, 1992.
\subsection{Bond distances}

Equlibrium pairwise distances are given by

\begin{equation}
    r_0 = r_i + r_j + r_\text{BO} + r_\text{EN}
\end{equation}

where $r_k$ is the covalant radius of atom $k$, $r_{BO}$ a correction based on
the bond order between the two atoms and $r_{EN}$ a correction for the
electronegativity.

\begin{equation}
    r_\text{BO} = -\lambda (r_i + r_j) \ln(n)
\end{equation}

where $\lambda = 0.1332$ and $n$ the bond order. For example, $n=1.5$ for aromatic bonds.

\begin{equation}
    r_\text{EN} = r_i r_j \frac{(\sqrt{\chi_i} - \sqrt{\chi_j})^2}
                               {\chi_i r_i \chi_j r_j}
\end{equation}

where $\chi_k$ is the GMP electronegativity of atom $k$. The energy is then a simple
harmonic

\begin{equation}
    E_{b} = \frac{k_{ij}}{2}(r - r_0)^2
\end{equation}

The derivative is

\begin{equation}
    \frac{\partial E}{\partial{X_{i, n}}} = k_{ij}
    \left(1 - \frac{r_0}{r}\right) (X_{i, n} - X_{j, n})
\end{equation}

where $X_{i, n}$ is the $n^\text{th}$ component of the Cartesian coordinate of
atom $i$ e.g. x, y, or z.


\subsection{Bond Force Constants}

Bond stretch force constants are defined as

\begin{equation}
    k_{ij} = 664.12 \frac{Z_i^* Z_j^*}{r_{ij}^3}
\end{equation}

where $Z_k^*$ is the effective atomic charges in units of $e$ and the energy in
kcal mol$^{-1}$.

\subsection{Angle Bends}

In general, UFF defines the energy of an angle bend as

\begin{equation}
    E_{\theta} = k_{ijk} \sum_{n=0}^m C_n \cos(n\theta)
\end{equation}

where for linear ($n=1$), trigonal-planar ($n=3$), square-planar ($n=4$) and
octahedral ($n=4$)

\begin{equation}
    E_{\theta}^\text{Type A} = \frac{k_{ijk}}{n^2} (1 - \cos(n\theta))
\end{equation}

the derivative is evaluated using sympy\footnote{\url{https://docs.sympy.org/}}

For other coordination envrionments with an equilibrum bond angle ($\theta_0$)

\begin{equation}
    E_{\theta}^\text{Type B} = k_{ijk}(C_0 + C_1\cos(\theta) + C_2\cos(2\theta))
\end{equation}
\begin{equation}
    C_2 = \frac{1}{4\sin^2(\theta_0)} \quad ;\quad C_1 = -4C_2\cos(\theta_0)
    \quad ; \quad C_0 = C_2(2\cos^2(\theta_0) + 1)
\end{equation}

\subsection{Angle Force Constants}

Angle force constants are defined as

\begin{equation}
    k_{ijk} = \beta \frac{Z_i^* Z_k^*}{r_{ik}^5} r_{ij}r_{jk}
    \left[r_{ij}r_{jk}(1-\cos^2(\theta_0)) - r_{ik}^2\cos(\theta_0) \right]
\end{equation}
\begin{equation}
    \beta = \frac{664.12}{r_{ij} r_{jk}}
\end{equation}

\clearpage
\subsection{Torisional Dihedrals}

For a sequence of bonded atoms $i-j-k-l$ the torsional energy is given by

\begin{equation}
    E_\phi = \frac{V_\phi}{2} [1 - \cos (n_\phi \phi_0)\cos(n_\phi \phi)]
\end{equation}

where $\phi$ is the torsinal angle, $V_\phi$ a force constant and $n_\phi$ the
multiplicity. The general cases are

\begin{enumerate}
    \item sp$^3$-sp$^3$: $n_\phi = 3; \phi_0 = 180^\circ$ unless $j, k$ are group 16
        atoms where $n_\phi = 2; \phi_0 = 90^\circ$.

    \item sp$^2$-sp$^3$: $n_\phi = 6; \phi_0 = 0^\circ$

    \item sp$^2$-sp$^2$: $n_\phi = 2; \phi_0 = 180^\circ$

    \item sp$^2$-sp$^2$-sp$^3$-X: $V_0 = 2.0; n_\phi=3; \phi_0 = 180^\circ$
\end{enumerate}

and

\begin{equation}
    V_{\text{sp}^3} = \sqrt{V_j V_k}
\end{equation}

where $V_m$ are tabulated values, while bonds containing for sp$^{2}$ centres

\begin{equation}
    V_{\text{sp}^2} = 5\sqrt{U_j U_k} (1+4.18\ln(n_{BO, jk}))
\end{equation}

where $U_m$ are values based on the period of the atom (indexed by $m$) and
$n_{BO, jk}$ is the value of the bond order between atoms $j$ and $k$.

Torsional potentials are only considered where the central bonds are main group
and the atoms non sp hybridised.

If either of the bond angles approaches $180^\circ$ then the potential is set to zero.

% Does it matter if it is non-smooth? Could be smoothed with e.g.
% w(x) = 1-0.5*((tanh(10*(x-3))+1) + (1-tanh(10*(x+3))))

\subsection{Inversion}

Trigonal(-pyramidal) geometeries are enforced using a cosine expansion in the
angle ($\gamma$) made the axis $c\text{-}k$ with the normal to the plane made by the
atoms $c, i, j$ where $c$ is the central atom and $i,j,k$ are bonded neighbours.

\begin{equation}
    E_\gamma = k_{cijk} (C_0 + C_1 \sin(\gamma) + C_2 \cos(2\gamma))
\end{equation}

where $k_{cijk}$ is a force constant and $C_n$ are constants defined by fitting
the function to a minimum at $\gamma_0$ and a maximum at the inversion point (for
group 15 central atoms).

As there are three potential axes ($c\text{-}m$) an average is taken over all three.


\subsection{Non-bonded: van der waals}

Dispersion + repulsive terms to the energy are given by a standard Lennard-Jones
12-6 potential

\begin{equation}
    E_\text{vdw} = D_{ij} \left[\left( \frac{\sigma_{ij}}{r_{ij}} \right)^{12}
                                -2 \left( \frac{\sigma_{ij}}{r_{ij}} \right)^6
                            \right]
\end{equation}

where $r_{ij}$ are the distance between atoms $i$ and $j$ and the distance
terms $\sigma_{ij}$ are generated using geometric mixing
$\sigma_{ij} = \sqrt{r_i r_j}$ and likewise for the well depth
$D_{ij} = \sqrt{D_i D_j}$ where $r_k, D_k$ are tabulated parameters.


\section{RB}
The theory here is based on T. Young et. al, ACIE, 2021.

\subsection{Bonds}
Bonds between atoms are described by harmonic terms

\begin{equation}
    E_{b} = \frac{k}{2}(r - r_0)^2
\end{equation}
where $k$ is constant for all bonds and $r_0$ is simply the sum of covalent
radii of atoms $i$ and $j$.

\subsection{Repulsion}
To simulate angle, torsional and repulsive non-bonded terms of a standard forcefields
a repulsion between all atom pairs is added.\footnote{This implementation adds
bonded pair exclusions.} Attractive dispersive terms are neglected.

\begin{equation}
    E_{r} = \frac{c}{r_{ij}^n}
\end{equation}
where $c$ is a fixed constant, $r_{ij}$ is the distance between atoms $i$ and
$j$ and $n$ is a flexible positive integer e.g. 2.


\end{document}
